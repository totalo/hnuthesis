\chapter{绪\quad 论}

\section{研究背景以及意义}

人类活动持续排放的温室气体如$\text{CO}_2$等导致了温室效应和全球变暖。尽管有越来越多的缓解气候变化的政策和措施,人类$\text{CO}_2$排放量平均增长率在2000-2014年期间每年还是达到了$2.6{\%}$,而在1970-2000年期间的年平均增长率却只有$1.72{\%}^{[1]}$。

一二三四五六七八九十一二三四五六七八九十一二三四五六七八九十一二三四五六七八九十一二三四五六七八九十

\section{$\text{CO}_2$捕获技术}

\subsection{概述}

目前,CCS过程被认为是最具前途的温室气体减排方法,该过程旨在将大型工业设施所排放的$\text{CO}_2$在进入大气之前进行捕获$^{[8 , 9]}$。这些技术涉及从燃料燃烧或工业过程中捕获$\text{CO}_2$,通过船舶或管道输送$\text{CO}_2$,将其永久储存在地质构造深处,或进行石油采集驱替提高石油采收率。

\subsubsection{第二小小节}
